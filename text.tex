\chapter{Úvod}
% v první kapitole není fancyhdr normálně, proto je potřeba ho nastavit:
\thispagestyle{fancy}

Lorem ipsum dolor sit amet, consectetur adipiscing elit. Proin vel nibh nisi. Vivamus aliquet vehicula dapibus. Ut vitae risus dignissim, consectetur nisi vitae, sollicitudin nibh. Integer sollicitudin viverra ipsum, a lobortis ipsum posuere malesuada. Suspendisse sed cursus diam. Maecenas cursus viverra pulvinar. Sed pretium tortor quis ultricies aliquet. Praesent consectetur, diam sit amet elementum imperdiet, nisl nisl aliquam metus, at maximus turpis eros eu diam. Ut lacinia aliquet tempor. Duis vitae nulla ex. Lorem ipsum dolor sit amet, consectetur adipiscing elit.

\section{První sekce}

Morbi at libero volutpat, fringilla lacus eget, aliquam nisl. Nam malesuada sodales fringilla. Nulla facilisi. Pellentesque facilisis lacinia nunc, quis egestas felis aliquam lobortis. Pellentesque ultricies ac arcu suscipit convallis. Cras porta quis metus quis elementum. 

\subsection{Subsekce}
Nullam placerat dignissim lorem pretium lacinia. Aenean consectetur, nisl ac rhoncus vehicula, ante tellus fringilla dui, et gravida nulla lacus nec neque. Aenean egestas fringilla dolor, nec tristique odio lacinia vel. Proin lacinia, enim sit amet commodo luctus, enim sem rutrum orci, eu condimentum quam nisi sit amet ex.

\begin{figure}
	\centering
	\includegraphics[width=0.5\textwidth]{figures/sample}
	\caption{Mandelbrotova množina \cite{knuthwebsite}}
\end{figure}

Lorem ipsum dolor sit amet, consectetur adipiscing elit. Proin vel nibh nisi. Vivamus aliquet vehicula dapibus. Ut vitae risus dignissim, consectetur nisi vitae, sollicitudin nibh. Integer sollicitudin viverra ipsum, a lobortis ipsum posuere malesuada. Suspendisse sed cursus diam. Maecenas cursus viverra pulvinar. Sed pretium tortor quis ultricies aliquet. Praesent consectetur, diam sit amet elementum imperdiet, nisl nisl aliquam metus, at maximus turpis eros eu diam. Ut lacinia aliquet tempor. Duis vitae nulla ex. Lorem ipsum dolor sit amet, consectetur adipiscing elit. \cite{einstein,latexcompanion}

\begin{equation} \label{eq:binomicka-veta}
(a+b)^n=\sum_{i=0}^{n}\binom{n}{i}a^{n-i}b^{i}
\end{equation}

\begin{table}
	\centering
	\caption{Ukázka číselných soustav}
	\begin{tabular}{|r|l|}
		\hline
		7C0 & šestnáctková \\
		3700 & osmičková \\
		11111000000 & dvojková \\
		\hline \hline
		1984 & desítková \\
		\hline
	\end{tabular}
\end{table}

Binomická věta je vidět v rovnici \ref{eq:binomicka-veta}, která je na straně \pageref{eq:binomicka-veta}.

Morbi at libero volutpat, fringilla lacus eget, aliquam nisl. Nam malesuada sodales fringilla. Nulla facilisi. Pellentesque facilisis lacinia nunc, quis egestas felis aliquam lobortis. Pellentesque ultricies ac arcu suscipit convallis. Cras porta quis metus quis elementum. Nullam placerat dignissim lorem pretium lacinia. Aenean consectetur, nisl ac rhoncus vehicula, ante tellus fringilla dui, et gravida nulla lacus nec neque. Aenean egestas fringilla dolor, nec tristique odio lacinia vel. Proin lacinia, enim sit amet commodo luctus, enim sem rutrum orci, eu condimentum quam nisi sit amet ex. 
