\usepackage[left=4cm,right=1.5cm,top=2.5cm,bottom=2.5cm,nohead]{geometry}
% pro debug
% \usepackage[left=4cm,right=1.5cm,top=2.5cm,bottom=2.5cm,nohead,showframe]{geometry}

\usepackage[T1]{fontenc}

% nastavíme Arial jako hlavní font. Tohle je podporováno jen compilery xelatex a lualatex
\usepackage{fontspec}
\setmainfont{Times New Roman}
\setsansfont{Arial}

% nastavení velikosti nadpisů a jejich fontu
% pozor, smíme použít maximálně subsection
\usepackage{anyfontsize}
\usepackage{titlesec}

\usepackage{url}
\DeclareUrlCommand\url{\def\UrlLeft{<}\def\UrlRight{>} \urlstyle{tt}}

\usepackage{multicol}

\titleformat{\chapter}
{\normalfont\sffamily\huge\bfseries}{\thechapter}{22pt}{}
%{\chaptertitlename\ \thechapter}{22pt}{\Huge}
\titleformat{\section}
{\normalfont\sffamily\fontsize{16pt}{19pt}\bfseries}
{\thesection}{1em}{}
\titleformat{\subsection}
{\normalfont\sffamily\fontsize{14pt}{17pt}\bfseries}
{\thesubsection}{1em}{}

% první paragraf po začátku sekce bude mít odsazenou první řádku
\usepackage{indentfirst}

% sekce 1.1.2 "Před i za názvem kapitoly, resp. podkapitoly následuje jedno až dvou řádková mezera."
\titlespacing{\chapter}{0pt}{*0}{40pt}
\titlespacing{\chapter}{0pt}{1.5em plus .5em minus .5em}{1.5em plus .5em minus .5em}
\titlespacing{\chapter}{0pt}{1.5em plus .5em minus .5em}{1.5em plus .5em minus .5em}

% Skoro Times New Roman pro pdflatex:
%\usepackage{mathptmx}

% vypne číslování rovnic ve formátu 3.2 a namísto toho čísluje globálně
\usepackage{chngcntr}
\counterwithout{equation}{subsection}
\counterwithout{equation}{section}
\counterwithout{equation}{chapter}
\counterwithout{figure}{subsection}
\counterwithout{figure}{section}
\counterwithout{figure}{chapter}
\counterwithout{table}{subsection}
\counterwithout{table}{section}
\counterwithout{table}{chapter}

\usepackage{fancyhdr}
\pagestyle{fancy}
\fancyhf{}
% odstrani caru v headeru, ktera je defaultne ve fancyhf
\renewcommand{\headrulewidth}{0pt}
% nastavi format cislovani stranek na - 3 -
\cfoot{\normalfont- \thepage \hspace{1pt} -}
% ^ tohle ovsem z nejakeho duvodu nefunguje na prvni strance, mozna proto, ze \maketitle vola \thispagestyle{plain} a to prejde i na dalsi stranku
% proto se po prvni \chapter musi zapsat prikaz \thispagestyle{fancy}

% vypne cislovani stranek v TOC
\addtocontents{toc}{\protect\thispagestyle{empty}}

\usepackage{setspace}
\setstretch{1.5}
% nastaví line-spacing na 1.5pt
% \renewcommand{\baselinestretch}{1.5}

% místo nadpisu Literatura
%\renewcommand\refname{Zdroje}
